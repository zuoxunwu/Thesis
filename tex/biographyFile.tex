%\justify

\hspace{2em} As far as he remembers, this guy has always wanted to be a physicist.
Xunwu Zuo was born into a family of hydraulic engineering background.
He enjoyed talking to his parents about dams and turbines since he was a little kid.
In retrospect, the first sense of "physics" Xunwu got was probably from a hardcover picture book he read at a Xinhua Bookstore.
The book starts with a picture of a galaxy cluster on the first page, 
with a corner circled leading to a zoomed-in picture of a local group on the second page.
The book kept zooming in page by page to stages like the milky way, the earth, cells, and molecules, all the way to quarks.
He was so into the book and thought it would be super cool to extend the book with some more pages.
He was lucky to have kept this interest in physics ever since.

Xunwu graduated with a B.S. in physics from the University of Science and Technology of China in 2011,
and then went to the University of Florida for his Ph.D.
He chose to work on experimental particle physics, 
which he thinks is the most meaningful area to make physics breakthroughs.
He stayed at the UF campus from 2015 to 2018, and at the CERN campus from 2018 to 2021.
Xunwu is glad to have made contributions to some important works of the CMS Collaboration.

Out of research, Xunwu has a wide range of hobbies.
He is not particularly good at any but does enjoy all of them.
He practices strength training regularly and tries new ideas in cooking every now and then.
In Florida, He enjoyed fencing, arranging and singing A Cappella music,
while in Geneva, he spent more time on hiking, bouldering, and skiing.
In general, Xunwu enjoys thinking about philosophy and gaining knowledge of any kind.
It makes him calm and content, which turned out to be precious during the 2020-2021 COVID pandemic.

Xunwu feels really lucky to be born into a loving family in a great era of humankind.
He wants to do some good things in return,
and continuing a career adding bricks to fundamental physics research is one of them.