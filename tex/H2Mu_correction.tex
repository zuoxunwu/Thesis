\chapter{Muon momentum correction and calibration} \label{chp:muon_corr}

This analysis aims to find a shape signal peak on top of a smooth background in the \mmm distribution.
It is of crucial importance to correct the mismeasurement in muon momentum scale and improve the momentum resolution.
The shaper muon peak, the better sensitivity of this analysis.
It is also crucial to reduce the differences in the muon momentum scale and resolution between data and simulation, 
so that there is no significant bias in the modeling of the signal.

Three sets of corrections are applied in this analysis: 
the \RochCorr~\cite{Bodek:2012id}, the recovery of the final state radiation (FSR) photons, and the \GeoFit.
The \RochCorr is a centrally provided correction (by CMS) which corrects 
the biases in the muon momentum resulted from the mismodeling of the detector alignment and the magnetic field. 
A brief description of the \RochCorr is given in Section~\ref{sec:Roch_corr}, 
while the technical details can be found in the Ref.~\cite{Bodek:2012id}.
The \FSR is a common practice in many CMS analyses which corrects the muon energy loss via FSR radiation.
The recovery scheme in this analysis is optimized specifically for the \hmm decay, which is described in Section~\ref{sec:fsr} and in more details in Ref.~\cite{}.
The \GeoFit is developed by the author in the context of the \hmm analysis and approved by the CMS collaboration.
It uses information of muon vertexing to correct the biases in the muon momentum of the mis-reconstructed muon tracks.
The development of the \GeoFit is described in details in Section~\ref{sec:GeoFit}.
The effects these three corrections focus on are orthogonal, and are even mutually exclusive between the \FSR and GeoFit.
In practice, the \RochCorr is applied to all muons, then each muon is surveyed for FSR photons.
If a FSR photon is found associated to the muon, the \FSR is applied, 
if not, the \GeoFit is applied to the muon.

The \RochCorr, \FSR, and \GeoFit are applied to both the data and the simulation.
The performance of the muon corrections is examined with the studies on the \zmm peak,
which is listed in details in Section~\ref{sec:muon_cal}.
These corrections fix all the known biases in the muon measurement, 
and ensure a per-mille-level agreement between the data and simulation. 

\section{Rochester correction} \label{sec:Roch_corr}

In reality, the CMS detector can have various imperfections, 
such as the misalignment of the detector components, and the uncertainties in the magnetic field.
Sometimes these imperfections are not correctly represented in the reconstruction software, 
and as a result, the reconstructed muons can be inaccurate.
These measurement biases are reflected as the dependences of the muon momemtum on its \eta, \phi ~coordinates, and its charge.
These dependences, in turn, smear the inclusive muon resolution and lead to sub-optimal physics results.

On the other hand, the detector simulation does not assume any of the imperfections,
which leads to differences in the detector response from the data, and eventually mismodelings (usually over-optimistic modelings) in the muon measurement. 
Therefore, the correction also needs to be applied to the simulation, which may not be the same as the correction on data.
Instead of using the reconstructed muons in the simulation as the reference for the correction,
the generated muon information is taken and smeared with some funtional forms to match the experimental resolution.
This provides a set of reference muons that is free from any of the reconstruction effects. 

The well-understood \zmm events are used to develop the \RochCorr. 
The idea of the correction is briefly summarized as follows:
\begin{itemize}
  \item For the data, reconstructed simulation (reco-sim) and reference simulation (ref-sim), 
        muons are divided into different \eta ~and \phi ~bins, separately for $\mu^{+}$ and $\mu^{-}$.
        In each bin, the $1/\pt$ distributions of the data and reco-sim are corrected so that the mean value of the distribution becomes the same as that in the ref-sim.
  \item The $1/\pt$ in reco-sim is usually narrower than that in data. 
        A smearing is applied to the reco-sim $1/\pt$ distribution so it matches the resolution in data.
  \item After the steps above, the \mmm in each bin may still be off from the expected distribution by some small amounts.
        The ratio between this offset and the nominal \PZ mass is applied to the muon \pt as a correction factor iteratively until the offset is minimized.
\end{itemize}
The \RochCorr removes the \mmm dependences on the muon \eta,~ \phi,~ and charge, 
as well as the \mmm resolution differences between the data and the simulation.
Details of the performance of the \RochCorr can be found in Section~\ref{sec:muon_cal}.


\section{FSR recovery} \label{sec:fsr}

In CMS, muons produced in pp collisions may radiate photons and lose energy, which is referred to as the final state radiation (FSR).
The radiate may carry substantial energy and lead to a underestimation of the muon momentum,
In an analysis that relies on the dimuon mass \mmm, the FSR may lead to two effects that degrade the sensitivity:
a loss of event acceptance, and a smearing of the \mmm resolution.
To mitigate these effects, some of the FSR photons can be identified can added back to the muon energy, which is called the \FSR.

The selection for the FSR photons is modified on top of the strategy developed in the CMS $\PH \to \PZ\PZ$ analyses~\cite{Sirunyan:2017exp, Sirunyan:2018qlb}.
The selection criteria is summarized as follows:
\begin{itemize}
  \item Photons with transverse energy $E^{\gamma}_{T} > $ 2 \GeV and $|\eta|<1.4$, $1.6<|\eta|<2.4$ are considered.
  \item The photon is required to be within the cone of $\Delta{}R<0.5$ around the closest muon to it which satisfies $\pt >$ 20 \GeV and $|\eta| < 2.4$.
  \item The photon is not identified as a bremsstrahlung photon associated with reconstructed electrons.
  \item The PF isolation of the photon in a cone of $\Delta{}R < 0.3$ should be less than 1.8, i.e. $\sum_{i}\pt^{i}(\Delta{}R(\gamma, i)<0.3)/\pt(\gamma) < 1.8$, 
        where $i$ iterates the PF objects around the photon other than the candidate muon.
  \item The separation between the photon and the muon satisfies $\Delta{}R(\mu, \gamma)/\pt^{2}(\gamma) < 0.012$.
  \item In order to suppress the energetic photons from the $\PH \to \PZ\gamma \to \mu\mu\gamma$ process,
        the \pt ratio between the photon and the muon is required to be less than 0.4, i.e. $\pt(\gamma)/\pt(\mu) < 0.4$.
  \item If multiple FSR photons are associated to a same muon, only the photon with the smallest $\Delta{}R(\mu, \gamma)/\pt^{2}(\gamma)$ is taken.
\end{itemize}

With this set of selection, about 3\% of the signal events are tagged with FSR photons.
The momentum of the FSR photons are added to the muon momentum, 
while photons themselves are ignored from the calculation of the muon isolation.
The \FSR significantly improves the \mmm reconstruction in the FSR tagged events, as shown in the left plot of Figure~\ref{fig:fsr_sig}.
Overall, as shown in the right plot of Figure~\ref{fig:fsr_sig}, the effect on the inclusive signal is a 3\% improvement on the \mmm resolution, 
and a 1.7\% increase in the total signal yield. 
The performance of \FSR is also validated with the \zmm events, shown in Figure~\ref{fig:fsr_val}.
A good agreement kept between the simulation and data with or without the \FSR. 
The \FSR is expected to perform the same way on data as on simulation, 
and no bias is introduced by the application of the \FSR. 

\begin{figure*}[!htb]
      \centering
      \captionsetup{justification=justified}
      \includegraphics[width=0.45\textwidth]{pics/muon_corr/FSR/FSRrecovery_FSRtagged.pdf}
      \includegraphics[width=0.45\textwidth]{pics/muon_corr/FSR/FSRrecovery_FullSignal.pdf}
      \caption{Taken from Ref.~\cite{}. 
               Performance of the \FSR in the simulated \hmm events. 
               The \mmm before and after the \FSR are shown for the events that contain at least one FSR photon (left),
               and for the inclusive signal events (right).}
      \label{fig:fsr_sig}
\end{figure*}

\begin{figure*}[!htb]
      \centering
      \captionsetup{justification=justified}
      \includegraphics[width=0.45\textwidth]{pics/muon_corr/FSR/FSRrecovery_Validation.pdf}
      \caption{Taken from Ref.~\cite{}.
               Performance of the \FSR in the \zmm events that contain FSR photons, in both data and simulation. 
               A good agreement between the simulation and the data is observed, before or after the correction.}
      \label{fig:fsr_val}
\end{figure*}


\section{GeoFit correction} \label{sec:GeoFit}

In CMS, the charged tracks are reconstructed from the hits in the tracker system.
The tracks may originate from the primary vertices (PV), which are vertices of the pp interactions,
or the secondary vertices, which are decay vertices of particles with a non-infinitesimal lifetime.
The track reconstruction does not assume any vertex information as the vertices are reconstructed as the intersections of groups of tracks.
This practice is essential for the studies of the \Pqb ~quarks, the \tau ~leptons, 
and any other long-lived particles that produce a vertex displaced from the colliding region in the transverse plane at the scale of a millemeter or above.
However, for the tracks which from prompt interactions, such as the decay of a \PZ boson, \PW boson, or \PH boson,
called the prompt tracks, even if they are expected not to have a visible displacement from the PV, 
a non-zero displacement may still appear in the reconstructed tracks due to the uncertainties in the track fit.
In other words, if a track is known to be prompt, the posterior information on the colliding position 
can be used to improve the quality of the track fit and in turn the measurement on the track momentum.

The study shown in this section reports the finding that this false displacement of the prompt tracks
has a strongly geometrical correlation with the mis-measurement of the track momemtum.
A simple analytic function can be derived from the geometry and can be verified by 
fitting the displacement vs the \pt bias in the simulated samples.
This analytic form is applied as the correction to the \pt and is therefore named the \GeoFit.
Section~\ref{sec:d0_geometry} explains the geometry of the track displacement and the correlation.
Section~\ref{sec:dev_geofit} describes the studies on the simulated samples in order to find the best fit parameters in that correlation.
The \GeoFit is developed using only the muon tracks in the context of the \hmm search.
It removes the dependence of the \mmm on the track displacement which leads to an 
improvement on the \mmm resolution of the combined signal ranging from 3\% to 10\%, depending on the data-taking period.
Details of the \GeoFit performance, along with various validation studies are shown in Section~\ref{sec:val_geofit}.
In addition, an alternative way to correct this \pt bias is to redo the track fit including the colliding vertex as an additional hit in the track, 
which should achieve a more fundamental correction at the cost of more computational resources.
A prelimimary set of study comparing the \GeoFit with the track re-fit shows the two correction methods give almost equivalent results,
detailed in Section~\ref{sec:track_refit}.



\subsection{Geometry of the track displacement}\label{sec:d0_geometry}

The displacement of a track from a vertex is usually measured as the impact parameters, $d_{xy}$ and $d_{z}$,
which are the signed distance between the vertex and its point of closest approach (PCA) on the track, in the transverse and longitudal directions.
In CMS, because most studies only care about the transverse impact parameter, 
the PCA is defined as the point on the 2D-projection of the track in the transverse plane that is the closest to the vertex.
(It is not necessarily the PCA in the 3D-space. The $d_{z}$ is calculated at the 3D-point corresponding to the 2D-PCA, rather than the 3D-PCA.)
As the $d_{z}$ is not used in our studies, the term "impace parameter", if not otherwise stated,
refers specifically to the transverse impact parameter $d_{xy}$, also denoted as d0.
The definition of d0 can be expressed as 
\begin{equation}\label{eq:d0_def}
  d0 = -x_{0} \cdot sin(\phi_{0}) + y_{0} \cdot cos(\phi_{0})
\end{equation}
where $(x_{0}, y_{0})$ is the coordinate of a point near the vertex in the frame where the vertex is at $(0,0)$, 
and the $\phi_{0}$ is the azimuthal angle of the track at that point.
A scheme for this definition is shown in Figure~\ref{fig:d0_def}.

\begin{figure*}[!htb]
      \centering
      \captionsetup{justification=justified}
      \includegraphics[width=0.70\textwidth]{pics/muon_corr/GeoFit/d0_def.pdf}
      \caption{Scheme of the d0 definition in CMS. The $(x_{0}, y_{0})$ is the coordinate of 
               a point near the vertex in the frame where the vertex is at $(0,0)$, 
               and the $\phi_{0}$ is the azimuthal angle of the track at that point.}
      \label{fig:d0_def}
\end{figure*}

\begin{figure*}[!htb]
      \centering
      \captionsetup{justification=justified}
      \includegraphics[width=\textwidth]{pics/muon_corr/GeoFit/d0_pt_geometry.pdf}
      \caption{Scheme of the track geometry in the transverse plane. 
               The blue lines show the geometry of the reconstructed track, 
               compared to the black lines which are the geometry of the true track.
               The difference between the blue track and black track is exaggerated in the scheme.
               The blue track and the black track must intersect at two points.
               $l$ is the distance between the two intersections, 
               and $x$ is the distance between the true vertex and the first intersection.
               $s$ is the distance between the centers of the two circular tracks.}
      \label{fig:d0_pt_scheme}
\end{figure*}

In the track geometry, illustrated in Figure~\ref{fig:d0_pt_scheme}, 
the reconstructed track is very close to, but slightly deviated from, the true track,
which leads to a small d0 between the reco track and the true vertex, 
and a small distance between the circular centers of the reco track and the true track.
The centers of the two tracks (circles) are labels as $O$ and $O'$ for the true track and the reco track,
and $s$ is the distance between $O$ and $O'$. 
The radii of the two tracks are $r$ and $r'$, with $\Delta{}r = r' - r$.
The two tracks must intersect at two points, labeled as point $M$ and $N$, 
and the distance between $M$ and $N$ denoted as $l$.
$\beta$ is half of the central angle spanned by the chord $l$ in the true track,
while $\alpha$ is the angle $\angle O'MO$.
The true vertex is denoted as $V$, with d0 as the impact parameter of the reco track to it.
while the PCA on the reco track is denoted as $P$.
$x$ is the distance between $M$ and $V$.
In addition, the radii of the tracks under study are at the scale of several tens of meters,
and the $\Delta{}r$ is expected to be much smaller than $r$.
Points $M$ and $N$ are expected to be around the coverage of CMS tracker system, which is about a meter.
Therefore $x$ and $l$ are expected to be much smaller than $r$ as well.
Finally, d0 scale of the tracks under study is about ten microns, which is much smaller than $x$, $l$, and $r$. 

In this setup, a few geometric relationships can be found between the different variables, listed as follows:
Since $x \ll r$, arc $\stackrel{\frown}{VM}$ and $\stackrel{\frown}{PM}$ can be viewed as line segments which are respectively perpendicular to $OM$ and $O'M$.
Therefore in triangle $\triangle VMP$, 
\begin{equation}\label{eq:d01}
      d0 = x \cdot sin\alpha
\end{equation}      
In triangle $\triangle O'MO$, the sine law gives
\begin{equation}\label{eq:d02}
      \frac{s}{sin\alpha} = \frac{r+\Delta{}r}{sin\beta}
\end{equation}
And in triangle $\triangle O'MK$, 
\begin{equation}\label{eq:d03}
      sin\beta = \frac{l/2}{r}
\end{equation}   
Then, using the Pythagorean theorem in both triangle $\triangle O'MK$ and triangle $\triangle OMK$, there is
\begin{equation}\label{eq:d04}
      s =  \sqrt{(r+\Delta{}r)^{2} - (l/2)^{2}} - \sqrt{r^{2} - (l/2)^{2}}
\end{equation}   
Combining Equation ~\ref{eq:d01} to ~\ref{eq:d04} and assuming $r \gg l$, one can get
\begin{equation}\label{eq:d05}
      d0 = \frac{xl}{2} \cdot \frac{\Delta{}r}{r^{2}}
\end{equation}  
Note that in CMS, under the 3.8T magnetic field, tracks follow 
\begin{equation}
    \pt ~(\text{in ~\GeV}) = 1.14 \cdot R ~(~\text{in meter})   
\end{equation}
Therefore we get
\begin{equation}\label{eq:d0_prop}
    d0 \propto \frac{\Delta{}\pt}{\pt^{2}}
\end{equation}

Now a quantitative relationship is extracted between d0 and \pt, but with one caveat:
the variables $x$ and $l$ in the scheme above may vary track by track, and are impossible to measure in real data,
meaning that coefficient in the proportionality is not a constant and Equation~\ref{eq:d0_prop} can be smeared.
Therefore, to follow up, studies are performed on simulated samples comparing the reconstructed \pt and the generated \pt of muon tracks.
Plots of $(\pt^{reco}-\pt^{gen})/ (\pt^{gen})^{2}$ vs d0 are made, 
to see whether the proportionality can be observed, after the smearing,
which is the topic of Section~\ref{sec:dev_geofit}.

Another remark needs to be made that in Figure~\ref{fig:d0_pt_scheme} the \pt mismeasurement is related to the relative position of the true vertex to the reconstructed track,
namely, if the true vertex is inside of the reco track, the \pt is over-estimated, 
while if the true vertex is outside of the reco track, the \pt is under-estimated. 
However, in the CMS definition of d0 shown in Figure~\ref{fig:d0_def}, the sign of d0 corresponds to an opposite relative position between the vertex and the track
for the positively charged muons and the negatively charged muons. 
A positive d0 value means the true vertex is inside of the reco track for the positive muons, but outside of the reco track for the negative cases.
Therefore in CMS convention the d0-\pt correlation is expected be reversed for different muon charges,
and in Section~\ref{sec:dev_geofit} studies are always performed evaluating d0 $\cdot$ charge rather than just d0.

\subsection{Development of GeoFit}\label{sec:dev_geofit}

The $(\pt^{reco}-\pt^{gen})/ (\pt^{gen})^{2}$ vs d0 plots are made with the following steps:
The values $\pt^{reco}$, $\pt^{gen}$, and d0 are extracted for each track in the simulated samples.
The distribution of $(\pt^{reco}-\pt^{gen})/ (\pt^{gen})^{2}$ is made for tracks in different d0 $\cdot$ charge bins.
The maximum position and the corresponding full-width-half-maximum (FWHM) is found for each 
fine-binned $(\pt^{reco}-\pt^{gen})/ (\pt^{gen})^{2}$ distribution and set as the value and uncertainty
of one data point in the plots in Figure~\ref{fig:pv_vs_bs_fits}.
The plots are then fit with analytic functions which are considered as the experimental realization of Equation~\ref{eq:d0_prop}.

In CMS, the colliding position can be measured as two different physics objects, the primary vertex (PV) and the beam spot (BS).
A primary vertex is a 3D-point which is compatible with several tracks in the same event.
Ususally several PVs are reconstructed per event, each of them considered as the position of a pp collision instance.
The beam spot, on the other hand, is actually a 3D-region in which most of the pp collisions happen,
and is reconstructed with all tracks in all the events in many consecutive luminosity sections.
The beam spot usually is around 10-20 \mum wide in the x, y directions, and about 7-9 \cm long in the z direction.

Both the PV and BS are reasonable representation of the colliding position and are useful in different cases.
The PV has a good z coordinate precision and can be used to judge if two tracks originate from the same interaction. 
But the position of the PV can be biased by the few energetic tracks associated to it, 
as the tracks are weighted by $\pt^{2}$ in the reconstruction of the PV.
The BS is wide-spread in z direction, but is less affected by individual tracks in its x, y coordinates.
To compare the two types of vertices, the d0 is measured regarding each of them, 
and examples of the $(\pt^{reco}-\pt^{gen})/ (\pt^{gen})^{2}$ vs d0 dependences are shown in Figure~\ref{fig:pv_vs_bs_fits}.
The dependence in the PV plot is not linear as the reconstructed PV is pulled towards the energetic muon tracks,
while the dependence in the BS plot follows a linear trend as predicted in Equation~\ref{eq:d0_def}.
Therefore, the BS is considered as the position of the true vertex in the rest of the study.

\begin{figure*}[!htb]
      \centering
      \captionsetup{justification=justified}
      \includegraphics[width=0.45\textwidth]{pics/muon_corr/GeoFit/fit_results/d0_pt_PV_eg.pdf}
      \includegraphics[width=0.45\textwidth]{pics/muon_corr/GeoFit/fit_results/d0_pt_BS_eg.pdf}
      \caption{Example plots showing the correlation between $(\pt^{reco}-\pt^{gen})/ (\pt^{gen})^{2}$ and d0 $\cdot$ charge.
               The vertices used for the d0 calculation are the PV (left) and the BS (right). 
               The PV plot shows a modulated dependence while the BS plot shows a linear shape as expected.
               Only barral tracks from 2016 data are shown as examples. 
               Plots of other |\eta| regions and other data-taking periods show simular behaviors.
               }
      \label{fig:pv_vs_bs_fits}
\end{figure*}

The $(\pt^{reco}-\pt^{gen})/ (\pt^{gen})^{2}$ vs d0 correlation is found to be different for different |\eta| regions and data-taking periods:
the different |\eta| regions are covered by different detector components, and there have been upgrades on the detector and the reconstruction algorithm between different data-taking periods.
Overall, the $(\pt^{reco}-\pt^{gen})/ (\pt^{gen})^{2}$ vs d0 correlation is evaluated by three years (2016, 2017, 2018) and 
three |\eta| regions (barrel, overlap, endcap), shown in Figure~\ref{fig:geofit_param_2016} for 2016, ~\ref{fig:geofit_param_2017} for 2017, and ~\ref{fig:geofit_param_2018} for 2018.
Each of the plot is fit with a linear function, whose best fit parameters are also shown in the plot.

\begin{figure*}[!htb]
      \centering
      \captionsetup{justification=justified}
      \includegraphics[width=0.32\textwidth]{pics/muon_corr/GeoFit/fit_results/2016_DY_eta_0_0p9_dRelPt2p0_Roch.png}
      \includegraphics[width=0.32\textwidth]{pics/muon_corr/GeoFit/fit_results/2016_DY_eta_0p9_1p7_dRelPt2p0_Roch.png}
      \includegraphics[width=0.32\textwidth]{pics/muon_corr/GeoFit/fit_results/2016_DY_eta_1p7_inf_dRelPt2p0_Roch.png}
      \caption{Plots for the $(\pt^{reco}-\pt^{gen})/ (\pt^{gen})^{2}$ vs d0 correlation in the 2016 \DY simulation, 
               and the linear fits to them. Muon tracks are divided into three different |\eta| regions:
               $|\eta| <$ 0.9 (left), 0.9 $< |\eta| <$ 1.7 (middle), and 1.7 $< |\eta|$ (right).
               }
      \label{fig:geofit_param_2016}
\end{figure*}

\begin{figure*}[!htb]
      \centering
      \captionsetup{justification=justified}
      \includegraphics[width=0.32\textwidth]{pics/muon_corr/GeoFit/fit_results/2017_DY_eta_0_0p9_dRelPt2p0_Roch.png}
      \includegraphics[width=0.32\textwidth]{pics/muon_corr/GeoFit/fit_results/2017_DY_eta_0p9_1p7_dRelPt2p0_Roch.png}
      \includegraphics[width=0.32\textwidth]{pics/muon_corr/GeoFit/fit_results/2017_DY_eta_1p7_inf_dRelPt2p0_Roch.png}
      \caption{Plots for the $(\pt^{reco}-\pt^{gen})/ (\pt^{gen})^{2}$ vs d0 correlation in the 2017 \DY simulation, 
               and the linear fits to them. Muon tracks are divided into three different |\eta| regions:
               $|\eta| <$ 0.9 (left), 0.9 $< |\eta| <$ 1.7 (middle), and 1.7 $< |\eta|$ (right).
               }
      \label{fig:geofit_param_2017}
\end{figure*}

\begin{figure*}[!htb]
      \centering
      \captionsetup{justification=justified}
      \includegraphics[width=0.32\textwidth]{pics/muon_corr/GeoFit/fit_results/2018_DY_eta_0_0p9_dRelPt2p0_Roch.png}
      \includegraphics[width=0.32\textwidth]{pics/muon_corr/GeoFit/fit_results/2018_DY_eta_0p9_1p7_dRelPt2p0_Roch.png}
      \includegraphics[width=0.32\textwidth]{pics/muon_corr/GeoFit/fit_results/2018_DY_eta_1p7_inf_dRelPt2p0_Roch.png}
      \caption{Plots for the $(\pt^{reco}-\pt^{gen})/ (\pt^{gen})^{2}$ vs d0 correlation in the 2018 \DY simulation, 
               and the linear fits to them. Muon tracks are divided into three different |\eta| regions:
               $|\eta| <$ 0.9 (left), 0.9 $< |\eta| <$ 1.7 (middle), and 1.7 $< |\eta|$ (right).
               }
      \label{fig:geofit_param_2018}
\end{figure*}

These fit results are applied as the analytical correction to the muon \pt, 
based on the d0, \pt, |\eta| and charge of the muon.
The correction is applied to all muon in the data and simulation in all categories in the \hmm analysis, 
unless the muon is tagged for \FSR.
The performance of this correction is detailed in Section~\ref{sec:perf_geofit}.

\subsection{Performance and validation}\label{sec:perf_geofit}

The \GeoFit removes the \pt dependence on d0, 
the overall effect of which on the \zmm peak is illustrated in Figure~\ref{fig:mucal_d0_run2}.
A clear trend in the \mmm is seen regarding to d0 before the \GeoFit, 
while no significant dependence remains after the \GeoFit.
As a side remark, Figure~\ref{fig:mucal_d0_run2} shows the mismeasurement in \mmm can be as large as 1.5 \GeV,
but in the data and simulation the distribution of the muon d0 is roughly a Gaussian shape with a standard deviation around 15 \mum.
So most of the events are near the center of the plots, and the size of the correction is not as exaggerated as the values at the tails. 

\begin{figure*}[!htb]
      \centering
      \captionsetup{justification=justified}
      \includegraphics[width=0.45\textwidth]{pics/muon_corr/GeoFit/performance/muP_d0_summary_mean.pdf}
      \includegraphics[width=0.45\textwidth]{pics/muon_corr/GeoFit/performance/muN_d0_summary_mean.pdf}
      \caption{Plots showing the \pt dependence on the d0 value with different stages of muon correction.
               The plots compare the \zmm peak the data and the simulation for three years (2016-2018) combined.
               All positively charged muons are put in the left plot and all negatively charged ones are put in the right plot.
               The \pt-d0 dependence is reversed for the positive and negative muons.
               }
      \label{fig:mucal_d0_run2}
\end{figure*}

Overall, the removal of the \pt-d0 dependence leads to an improvement on the inclusive \mmm resolution.
This improvement is different for different processes depending on their kinematic profiles in \pt and |\eta|.
Figure~\ref{fig:geofit_sigs} shows the improvement on \mmm resolution in the four main expected signal modes, \ggH, \qqH, \VH, and \ttH.
The relative improvements on \mmm resolution for \ggH, \qqH, \VH, and \ttH modes are, 
respectively, 6.1\%, 7.8\%, 8.0\%, and 9.8\%.
This improvement on signal resolution translates into about 5\% improvment on the significance of the inclusive \hmm analysis.

\begin{figure*}[!htb]
      \centering
      \captionsetup{justification=justified}
      \includegraphics[width=0.45\textwidth]{pics/muon_corr/GeoFit/performance/ggHVBF.pdf}
      \includegraphics[width=0.45\textwidth]{pics/muon_corr/GeoFit/performance/VHttH.pdf}
      \caption{Taken from Ref.~\cite{cmscollaboration2020evidence}.
               Plots showing the \GeoFit improvement on the four main \hmm signal modes, 
               \ggH and \qqH plotted on the left and \VH and \ttH plotted on the right.
               The plots are made combining the expected signal in all three years of data-taking (2016-2018).
               The relative improvements on \mmm resolution for \ggH, \qqH, \VH, and \ttH modes are, respectively,
               6.1\%, 7.8\%, 8.0\%, and 9.8\%.
               }
      \label{fig:geofit_sigs}
\end{figure*}



\subsection{GeoFit vs track re-fit}\label{sec:track_refit}


\section{Muon calibration results} \label{sec:muon_cal}

The \zmm is a well-understood process with a mass scale not far from the Higgs boson and much more number of events in CMS.
It is therefore used as a candle to monitor the performance of the \RochCorr and the \GeoFit, 
and validate that these corrections do not introduce new biases.
In this study, the distribution of the \mmm is plotted in different bins of some dimuon kinematic variables.
The \mmm distributions are fit with a Voigtian + Exponential function, 
the Voigtian part being a convoluiton of a Breit-Wigner function and a Gaussian function.
The parameters mean mass from the Breit-Wigner part and standard deviation from the Gaussian part are 
taken as the mean value and the experimental resolution of the \mmm distribution,
and are plotted against the dimuon kinematic variable of interest to check for potential trends.

The calibration plots are made by year as the corrections are provided by year.
Events containing \FSR are removed from this study as it is a separate effect.
Different variables are tested in the Figures listed:

From these plots, it can be concluded that all the known biases in muon \pt are removed and no new bias has been introduced.
The \RochCorr and \GeoFit correct orthogonal effects, and do not interfere the performance of each other.
After the corrections, a per-mille level agreement is achieved between data and simulation in the \mmm value,
while the agreement in \mmm resolution is about a few percent. 

\begin{figure*}[!htb]
      \centering
      \captionsetup{justification=justified}
      \includegraphics[width=0.32\textwidth]{pics/muon_corr/muon_cal/2016/muP_eta_summary_mean.pdf}
      \includegraphics[width=0.32\textwidth]{pics/muon_corr/muon_cal/2017/muP_eta_summary_mean.pdf}
      \includegraphics[width=0.32\textwidth]{pics/muon_corr/muon_cal/2018/muP_eta_summary_mean.pdf}
      \includegraphics[width=0.32\textwidth]{pics/muon_corr/muon_cal/2016/muP_eta_summary_reso.pdf}
      \includegraphics[width=0.32\textwidth]{pics/muon_corr/muon_cal/2017/muP_eta_summary_reso.pdf}
      \includegraphics[width=0.32\textwidth]{pics/muon_corr/muon_cal/2018/muP_eta_summary_reso.pdf}
      \caption{Muon calibration plots vs $\eta(\mu^{+})$, for 2016 (left column), 2017 (middle column) and 2018 (right column).
               The top row shows the mean value of the \mmm while the bottom row shows its experimental resolution.}
      \label{fig:mucal_muP_eta}
\end{figure*}


\begin{figure*}[!htb]
      \centering
      \captionsetup{justification=justified}
      \includegraphics[width=0.32\textwidth]{pics/muon_corr/muon_cal/2016/muP_phi_summary_mean.pdf}
      \includegraphics[width=0.32\textwidth]{pics/muon_corr/muon_cal/2017/muP_phi_summary_mean.pdf}
      \includegraphics[width=0.32\textwidth]{pics/muon_corr/muon_cal/2018/muP_phi_summary_mean.pdf}
      \includegraphics[width=0.32\textwidth]{pics/muon_corr/muon_cal/2016/muP_phi_summary_reso.pdf}
      \includegraphics[width=0.32\textwidth]{pics/muon_corr/muon_cal/2017/muP_phi_summary_reso.pdf}
      \includegraphics[width=0.32\textwidth]{pics/muon_corr/muon_cal/2018/muP_phi_summary_reso.pdf}
      \caption{Muon calibration plots vs $\phi(\mu^{+})$, for 2016 (left column), 2017 (middle column) and 2018 (right column).
               The top row shows the mean value of the \mmm while the bottom row shows its experimental resolution.}
      \label{fig:mucal_muP_phi}
\end{figure*}


\begin{figure*}[!htb]
      \centering
      \captionsetup{justification=justified}
      \includegraphics[width=0.32\textwidth]{pics/muon_corr/muon_cal/2016/muN_phi_summary_mean.pdf}
      \includegraphics[width=0.32\textwidth]{pics/muon_corr/muon_cal/2017/muN_phi_summary_mean.pdf}
      \includegraphics[width=0.32\textwidth]{pics/muon_corr/muon_cal/2018/muN_phi_summary_mean.pdf}
      \includegraphics[width=0.32\textwidth]{pics/muon_corr/muon_cal/2016/muN_phi_summary_reso.pdf}
      \includegraphics[width=0.32\textwidth]{pics/muon_corr/muon_cal/2017/muN_phi_summary_reso.pdf}
      \includegraphics[width=0.32\textwidth]{pics/muon_corr/muon_cal/2018/muN_phi_summary_reso.pdf}
      \caption{Muon calibration plots vs $\phi(\mu^{-})$, for 2016 (left column), 2017 (middle column) and 2018 (right column).
               The top row shows the mean value of the \mmm while the bottom row shows its experimental resolution.}
      \label{fig:mucal_muN_phi}
\end{figure*}


\begin{figure*}[!htb]
      \centering
      \captionsetup{justification=justified}
      \includegraphics[width=0.32\textwidth]{pics/muon_corr/muon_cal/2016/dimu_pt_summary_mean.pdf}
      \includegraphics[width=0.32\textwidth]{pics/muon_corr/muon_cal/2017/dimu_pt_summary_mean.pdf}
      \includegraphics[width=0.32\textwidth]{pics/muon_corr/muon_cal/2018/dimu_pt_summary_mean.pdf}
      \includegraphics[width=0.32\textwidth]{pics/muon_corr/muon_cal/2016/dimu_pt_summary_reso.pdf}
      \includegraphics[width=0.32\textwidth]{pics/muon_corr/muon_cal/2017/dimu_pt_summary_reso.pdf}
      \includegraphics[width=0.32\textwidth]{pics/muon_corr/muon_cal/2018/dimu_pt_summary_reso.pdf}
      \caption{Muon calibration plots vs $\pt(\mu\mu)$, for 2016 (left column), 2017 (middle column) and 2018 (right column).
               The top row shows the mean value of the \mmm while the bottom row shows its experimental resolution.}
      \label{fig:mucal_dimu_pt}
\end{figure*}


\begin{figure*}[!htb]
      \centering
      \captionsetup{justification=justified}
      \includegraphics[width=0.32\textwidth]{pics/muon_corr/muon_cal/2016/dimu_eta_summary_mean.pdf}
      \includegraphics[width=0.32\textwidth]{pics/muon_corr/muon_cal/2017/dimu_eta_summary_mean.pdf}
      \includegraphics[width=0.32\textwidth]{pics/muon_corr/muon_cal/2018/dimu_eta_summary_mean.pdf}
      \includegraphics[width=0.32\textwidth]{pics/muon_corr/muon_cal/2016/dimu_eta_summary_reso.pdf}
      \includegraphics[width=0.32\textwidth]{pics/muon_corr/muon_cal/2017/dimu_eta_summary_reso.pdf}
      \includegraphics[width=0.32\textwidth]{pics/muon_corr/muon_cal/2018/dimu_eta_summary_reso.pdf}
      \caption{Muon calibration plots vs $\eta(\mu\mu)$, for 2016 (left column), 2017 (middle column) and 2018 (right column).
               The top row shows the mean value of the \mmm while the bottom row shows its experimental resolution.}
      \label{fig:mucal_dimu_pt}
\end{figure*}


\begin{figure*}[!htb]
      \centering
      \captionsetup{justification=justified}
      \includegraphics[width=0.32\textwidth]{pics/muon_corr/muon_cal/2016/muP_d0_rebin_summary_mean.pdf}
      \includegraphics[width=0.32\textwidth]{pics/muon_corr/muon_cal/2017/muP_d0_rebin_summary_mean.pdf}
      \includegraphics[width=0.32\textwidth]{pics/muon_corr/muon_cal/2018/muP_d0_rebin_summary_mean.pdf}
      \includegraphics[width=0.32\textwidth]{pics/muon_corr/muon_cal/2016/muP_d0_rebin_summary_reso.pdf}
      \includegraphics[width=0.32\textwidth]{pics/muon_corr/muon_cal/2017/muP_d0_rebin_summary_reso.pdf}
      \includegraphics[width=0.32\textwidth]{pics/muon_corr/muon_cal/2018/muP_d0_rebin_summary_reso.pdf}
      \caption{Muon calibration plots vs $d0(\mu^{+})$, for 2016 (left column), 2017 (middle column) and 2018 (right column).
               The top row shows the mean value of the \mmm while the bottom row shows its experimental resolution.}
      \label{fig:mucal_muP_d0}
\end{figure*}


\begin{figure*}[!htb]
      \centering
      \captionsetup{justification=justified}
      \includegraphics[width=0.32\textwidth]{pics/muon_corr/muon_cal/2016/muN_d0_rebin_summary_mean.pdf}
      \includegraphics[width=0.32\textwidth]{pics/muon_corr/muon_cal/2017/muN_d0_rebin_summary_mean.pdf}
      \includegraphics[width=0.32\textwidth]{pics/muon_corr/muon_cal/2018/muN_d0_rebin_summary_mean.pdf}
      \includegraphics[width=0.32\textwidth]{pics/muon_corr/muon_cal/2016/muN_d0_rebin_summary_reso.pdf}
      \includegraphics[width=0.32\textwidth]{pics/muon_corr/muon_cal/2017/muN_d0_rebin_summary_reso.pdf}
      \includegraphics[width=0.32\textwidth]{pics/muon_corr/muon_cal/2018/muN_d0_rebin_summary_reso.pdf}
      \caption{Muon calibration plots vs $d0(\mu^{-})$, for 2016 (left column), 2017 (middle column) and 2018 (right column).
               The top row shows the mean value of the \mmm while the bottom row shows its experimental resolution.}
      \label{fig:mucal_muP_d0}
\end{figure*}